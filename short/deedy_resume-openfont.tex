%%%%%%%%%%%%%%%%%%%%%%%%%%%%%%%%%%%%%%%
% Deedy - One Page Two Column Resume
% LaTeX Template
% Version 1.1 (30/4/2014)
%
% Original author:
% Debarghya Das (http://debarghyadas.com)
%
% Original repository:
% https://github.com/deedydas/Deedy-Resume
%
% IMPORTANT: THIS TEMPLATE NEEDS TO BE COMPILED WITH XeLaTeX
%
% This template uses several fonts not included with Windows/Linux by
% default. If you get compilation errors saying a font is missing, find the line
% on which the font is used and either change it to a font included with your
% operating system or comment the line out to use the default font.
% 
%%%%%%%%%%%%%%%%%%%%%%%%%%%%%%%%%%%%%%
% 
% TODO:
% 1. Integrate biber/bibtex for article citation under publications.
% 2. Figure out a smoother way for the document to flow onto the next page.
% 3. Add styling information for a "Projects/Hacks" section.
% 4. Add location/address information
% 5. Merge OpenFont and MacFonts as a single sty with options.
% 
%%%%%%%%%%%%%%%%%%%%%%%%%%%%%%%%%%%%%%
%
% CHANGELOG:
% v1.1:
% 1. Fixed several compilation bugs with \renewcommand
% 2. Got Open-source fonts (Windows/Linux support)
% 3. Added Last Updated
% 4. Move Title styling into .sty
% 5. Commented .sty file.
%
%%%%%%%%%%%%%%%%%%%%%%%%%%%%%%%%%%%%%%%
%
% Known Issues:
% 1. Overflows onto second page if any column's contents are more than the
% vertical limit
% 2. Hacky space on the first bullet point on the second column.
%
%%%%%%%%%%%%%%%%%%%%%%%%%%%%%%%%%%%%%%

\documentclass[]{deedy-resume-openfont}

\begin{document}

%%%%%%%%%%%%%%%%%%%%%%%%%%%%%%%%%%%%%%
%
%     LAST UPDATED DATE
%
%%%%%%%%%%%%%%%%%%%%%%%%%%%%%%%%%%%%%%
%\lastupdated

%%%%%%%%%%%%%%%%%%%%%%%%%%%%%%%%%%%%%%
%
%     TITLE NAME
%
%%%%%%%%%%%%%%%%%%%%%%%%%%%%%%%%%%%%%%


\namesection{Pablo}{Martínez González}{Excited about Deep Learning, Computer Vision and Computer Graphics}{\cvscholar{d7F3nPUAAAAJ} \textbullet{} \cvorcid{0000-0001-6037-9815} \textbullet{} \cvlinkedin{pablo-martinez-gonzalez} \textbullet{} \cvgithub{PabloMzGz} \textbullet{} \cvemail{pmartinez@dtic.ua.es} \textbullet{} \cvemailscnd{mzgz.pablo@gmail.com}}{face.png}
 


%%%%%%%%%%%%%%%%%%%%%%%%%%%%%%%%%%%%%%
%
%     COLUMN ONE
%
%%%%%%%%%%%%%%%%%%%%%%%%%%%%%%%%%%%%%%

\begin{minipage}[t]{0.495\textwidth} 

%%%%%%%%%%%%%%%%%%%%%%%%%%%%%%%%%%%%%%
%     EDUCATION
%%%%%%%%%%%%%%%%%%%%%%%%%%%%%%%%%%%%%%

%%%%%%%%%%%%%%%%%%%%%%%%%%%%%%%%%%%%%%
%     EXPERIENCE
%%%%%%%%%%%%%%%%%%%%%%%%%%%%%%%%%%%%%%

\section{Work/Research Experience}

\runsubsection{University of Alicante} \\
\descript{Research Engineer and Faculty}
\location{Alicante, Spain. (2018 - Ongoing) 5 years}
\vspace{\topsep} % Hacky fix for awkward extra vertical space
\begin{tightemize}
	\item Full time research engineer as a consequence of my PhD grant. It includes the development of my PhD thesis and teaching hours to bachelor students, among others.
\end{tightemize}

\sectionsep

\runsubsection{Technische Universität Wien (TU Wien)} \\
\descript{Visiting Scholar}
\location{Vienna, Austria. (2021) 3 months}
%\vspace{\topsep} % Hacky fix for awkward extra vertical space
\begin{tightemize}
	\item Proposed a deep learning model based on 6D object pose estimators for predicting suitable grasping points and areas on objects. Collaborated with the Vision for Robotics research group.
\end{tightemize}

\sectionsep

%\runsubsection{Research Intern} \\
%\descript{University of Alicante}
%\location{(2018) 9 months}
%\vspace{\topsep} % Hacky fix for awkward extra vertical space
%\begin{tightemize}
%	\item \href{https://ieeexplore.ieee.org/stamp/stamp.jsp?tp=&arnumber=8594495}{\addfontfeature{Color=blue} Paper (IROS 2018) -> \textit{The RobotriX: An eXtremely Photorealistic and Very-Large-Scale Indoor Dataset of Sequences with Robot Trajectories and Interactions}}. \href{https://www.youtube.com/watch?v=CiRc5tCtCak}{\addfontfeature{Color=blue} Video}
%	\item Development of the tool for generating synthetic and photorealistic image data for deep learning algorithms.
%\end{tightemize}
%
%\sectionsep

\runsubsection{Gestión Tributaria Territorial (GTT)}
\\
\descript{Junior Programmer}
\location{Alicante, Spain. (2016 - 2017) 1 year}
\begin{tightemize}
	\item Worked on maintaining and extending a web platform (C\#) and its database architecture (SQL, PL/SQL). This platform is used by public organisms.
\end{tightemize}


\sectionsep

\section{Education} 

\subsection{University of Alicante}
\descript{PhD in Deep Learning and Computer Vision}
\location{Alicante, Spain. 2018 - 2023 | FPU Grant} 
\href{http://rua.ua.es/dspace/handle/10045/133830}{\addfontfeature{Color=blue} PhD thesis} related to 6D object pose estimation with deep learning techniques, and trained with synthetic data.

\sectionsep

\descript{B.Sc. in Computer Science}
\location{Alicante, Spain. 2011 - 2015 | 9.18/10 GPA}
High Academic Performance Group \\
B.Sc. Thesis: Privacy and security when accessing cloud data.\\
Specialised on data mining, computer vision, robotics and AI.\\
Extraordinary award for best academic record.

\sectionsep

\descript{B.Sc. in Physics}
\location{Alicante, Spain. 2018 - Ongoing}

\sectionsep

\subsection{Rey Juan Carlos University}
\descript{M.Sc. in Computer Graphics, Games and VR}
\location{Madrid, Spain. 2015 - 2017 | 8.28/10 GPA} 
M.Sc. Thesis: TempleOps: An online multiplayer third-person shooting game with random map generation (VG development).

\sectionsep

\subsection{University of Salford}
\descript{Erasmus Intensive Program: Big Data}
\location{Manchester, UK. 2014}

\sectionsep

%%%%%%%%%%%%%%%%%%%%%%%%%%%%%%%%%%%%%%
%     COURSEWORK
%%%%%%%%%%%%%%%%%%%%%%%%%%%%%%%%%%%%%%

%\section{Courses}
%\subsection{Udacity}
%
%\begin{tabular}{ll}
%	2018 & \href{https://www.udacity.com/course/deep-learning-pytorch--ud188}{\addfontfeature{Color=blue} Intro to Deep Learning with PyTorch}
%\end{tabular}
%
%\sectionsep
%
%\subsection{Coursera}
%
%\begin{tabular}{ll}
%	2018 & \href{https://www.coursera.org/specializations/deep-learning}{\addfontfeature{Color=blue} Deep Learning Specialization (5 courses)} \\
%	2013 & \href{https://www.coursera.org/learn/progfun1}{\addfontfeature{Color=blue} Functional Programming Principles in Scala}
%\end{tabular}
%
%\sectionsep
%
%\subsection{Workshops and Summer Schools}
%
%\begin{tabular}{ll}
%	2018 & \href{https://pumps.bsc.es/2018/front-page-content}{\addfontfeature{Color=blue} PUMPS+AI Summer School} \\
%	2013 & \href{http://jgpu.dtic.ua.es}{\addfontfeature{Color=blue}Workshop on GPGPU programming} \\ 
%	2013 & \href{https://web.ua.es/en/verano/2013/campus/curso-de-programacion-de-aplicaciones-cientificas-y-de-vision-por-computador-sobre-procesadores-graficos.html}{\addfontfeature{Color=blue} Summer School on GPGPU programming} \\ 
%	2013 & \href{http://www.dccia.ua.es/OpenGL/es/}{\addfontfeature{Color=blue}OpenGL in depth} \\
%\end{tabular}

\sectionsep




%%%%%%%%%%%%%%%%%%%%%%%%%%%%%%%%%%%%%%
%     LANGUAGES
%%%%%%%%%%%%%%%%%%%%%%%%%%%%%%%%%%%%%%

%\section{Languages}
%\begin{tabular}{ll}
%	English & Professional working proficiency (C1)\\
%	Spanish & Native proficiency \\ 
%	Catalan & Professional working proficiency (C1)\\ 
%\end{tabular}

%%%%%%%%%%%%%%%%%%%%%%%%%%%%%%%%%%%%%%
%
%     COLUMN TWO
%
%%%%%%%%%%%%%%%%%%%%%%%%%%%%%%%%%%%%%%

\end{minipage} 
\hfill
\begin{minipage}[t]{0.495\textwidth} 


%%%%%%%%%%%%%%%%%%%%%%%%%%%%%%%%%%%%%%
%     PUBLICATIONS
%%%%%%%%%%%%%%%%%%%%%%%%%%%%%%%%%%%%%%

\section{Publications}

\subsection{Journals}
\vspace{\topsep}
\begin{tightemize}
\item \publication{S. Oprea et al.}{A review on deep learning techniques for video prediction}{PAMI}{2020}{https://ieeexplore.ieee.org/abstract/document/9294028}

\item \publication{P. Martinez-Gonzalez et al.}{UnrealROX: An eXtremely Photorealistic Virtual Reality Environment for Robotics Simulations and Synthetic Data Generation}{VirtualReality}{2019}{https://link.springer.com/article/10.1007/s10055-019-00399-5}

\item \publication{S. Oprea et al.}{A visually realistic grasping system for object manipulation and interaction in virtual reality environments}{Computer\&Graphics}{2019}{https://www.sciencedirect.com/science/article/pii/S0097849319301098}

\item \publication{A. Garcia-Garcia et al.}{A survey on deep learning techniques for image and video semantic segmentation}{ASOC}{2018}{https://www.sciencedirect.com/science/article/pii/S1568494618302813}

\item \publication{H. Mora-Mora et al.}{Computational analysis of distance operators for the iterative closest point algorithm}{PloS One}{2016}{http://journals.plos.org/plosone/article?id=10.1371/journal.pone.0164694}
\end{tightemize}

\vspace{\topsep}

\subsection{Conferences}
\vspace{\topsep}
\begin{tightemize}
\item \publication{P. Martinez-Gonzalez et al.}{Synthetic contact maps to predict grasp regions on objects}{IJCNN}{2022}{https://doi.org/10.1109/IJCNN55064.2022.9892548}

\item \publication{P. Martinez-Gonzalez et al.}{UnrealROX+: An Improved Tool for Acquiring Synthetic Data from Virtual 3D Environments}{IJCNN}{2021}{https://ieeexplore.ieee.org/document/9534447}

\item \publication{S. Oprea et al.}{H-GAN: the power of GANs in your Hands}{IJCNN}{2021}{https://ieeexplore.ieee.org/abstract/document/9534144}

\item \publication{A. Garcia-Garcia et al.}{The RobotriX: An eXtremely Photorealistic and Very-Large-Scale Indoor Dataset of Sequences with Robot Trajectories and Interactions}{IROS}{2018}{https://ieeexplore.ieee.org/stamp/stamp.jsp?tp=\&arnumber=8594495}

%\item \publication{H. Mora-Mora et al.}{Efficient matching for the Iterative Closest Point algorithm by using low cost distance metrics}{AMCME}{2015}{http://www.inase.org/library/2015/barcelona/bypaper/AMCME/AMCME-06.pdf}
\end{tightemize}

\sectionsep

\section{Grants and awards}

\begin{tabular}{ll}
2018 & Spanish FPU Grant for PhD Studies \\
2015 & BSc in Computer Science extraordinary \\
& award for best academic record.  \\
\end{tabular}

\sectionsep

%%%%%%%%%%%%%%%%%%%%%%%%%%%%%%%%%%%%%%
%     SKILLS
%%%%%%%%%%%%%%%%%%%%%%%%%%%%%%%%%%%%%%

\section{Skills}
\descript{Focused on}
Deep Learning, Computer Vision and Virtual Reality \\
\sectionsep
\descript{Also interested in}
GPGPU programming, 3D computer graphics and Physics \\
\sectionsep
\descript{Tools and libraries}
UE4 \textbullet{} Unity \textbullet{} PyTorch \textbullet{} Tensorflow \textbullet{} Git \\
\sectionsep
\descript{Programming}
Python \textbullet{} C and C++ \textbullet{} CUDA \textbullet{} \LaTeX\ \textbullet{} Java \textbullet{} Matlab \\


\end{minipage} 
\end{document} \documentclass[]{article}